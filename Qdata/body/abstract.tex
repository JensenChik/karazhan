\begin{abstract}
循环神经网络是一种强有力的学习方法,它们可以灵活地使用上下文信息,并且对数据的局部变形具有较强的鲁棒性。在序列的标注这个任务上,输入的序列被转换成流式的标签,循环神经网络的特性使得它们十分适用于这个任务。LSTM(Long short-term memory)是一种非常有前景的循环结构,它可以在输入和输出间构建持久的延迟,因此可以保留很长的上下文信息。这篇文章的关注于以下几点内容,首先我们讨论目前在有监督序列标注这个任务上取得最好效果的模型---循环神经网络,随后会讨论一个特殊的结构,也就是LSTM。文章的的主要贡献在于(1)提出一种新的输出层形式,使得循环网络可以在输入和输出映射关系未知的情况下直接进行训练(2)将LSTM扩展到高维数据,例如图像或视频序列。我们将会展示语音识别、在线的和离线的手写数字识别、关键词标注、图像分割及分类的实验效果,用以说明循环神经网络对比其他时序算法(比如HMM)的优点。
\end{abstract}